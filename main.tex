\documentclass{report}
\usepackage[utf8]{inputenc}
\usepackage{graphics}
\graphicspath{ {./images/} }
\usepackage{multicol}

% width,height
\usepackage[a4paper, total={7in, 10in}]{geometry}

\title{IoT in Agriculture}
\author{Kasyap Velampalli}
\date{\today}

\begin{document}

    \begin{titlepage}
    \centering
        \vspace*{2cm}
        \Huge
        \textbf{IoT in Agriculture}
        
        \vspace*{0.6cm}
        \Large
        \textit{Research Paper Summary}
        
        \normalsize
        \vspace*{1.5cm}
        Kasyap Velampalli\\
        \vspace{0.2cm}
        21011101141\\
        \vspace{0.2cm}
        AI-DS B\\
        
     %\includegraphics[scale=0.5]{logo}
        
        \includegraphics{logo3}\\
        Computer Science and Engineering\\
        Shiv Nadar University, Chennai\\
        21st January 2023
        \vspace*{1cm}
    
\end{titlepage}

    
    \begin{center}
        \section*{IoT in Agriculture}
    \end{center}
\setlength{\columnsep}{1.0cm}
    \large
    \section{Summary}
    IoT smart agriculture products are designed to help monitor crop fields using sensors
and by automating irrigation systems. As a result, farmers and associated brands
can easily monitor the field conditions from anywhere without any hassle.\\
    
    \\
    
   IoT in agriculture uses robots, drones, remote sensors, and computer imaging combined with continuously progressing machine learning and analytical tools for monitoring crops, surveying, and mapping the fields, and providing data to farmers for rational farm management plans to save both time and money.



    
\begin{multicols}{1}    
    \section*{Robotics in agriculture}
    %our understanding on what the authors have contributed or raised as comments regarding the topic.
    
Since the industrial revolution in the 1800s, automation got more advanced to
efficiently handle sophisticated tasks and increase production. With increasing
demands and shortage of labor across the globe, agriculture robots or commonly
known as Agribots are starting to gain attention among farmers. Crop production
decreased by an estimated 213 crores approx (3.1 billion) a year due to labor
shortages in the USA alone. Recent advancements in sensors and AI technology that
lets machines train on their surroundings have made agrobots more notable. We are
still in the early stages of an ag-robotics revolution, harnessing the full potential of
the Internet of Things in agriculture, with most of the products still in early trial phases
and R&D mode
    
    \begin{enumerate}
        \item These smart Agri robots use digital image processing to look through the images of
weeds in their database to detect similarities with crops and weed out or spray them
directly with their robotic arms. With an increasing number of plants becoming
resistant to pesticides they are a boon to the environment and also to farmers who
used to spread the pesticides throughout the farm. 
        \item As remote-controlled toy cars are enabled with a controller, tractors and heavy
plowing equipment can be run automatically from the comfort of home through GPS.
These integrated automatic machines are highly accurate and self-adjust when they
detect differences in terrains, simplifying labor-intensive tasks.
        \item Utilizing agribots to pick crops is solving the problem of labor shortages. Working the
delicate process of picking fruits and vegetables these innovative machines can
operate 24/7. A combination of image processing and robotic arms is used by these
machines to determine the fruits to pick hence controlling the quality. .
        \item Robots can perform dreaded manual labor tasks working alongside the labors. They
can lift heavy materials and perform tasks like plant spacing with high accuracy,
therefore optimizing the space and plant quality and reducing production costs.
    \end{enumerate}
    

    \section*{My views about this paper}
    %Your views on the topic, along with what you envision the future of AI would be in the topic discussed in the paper.
   The paper has given us an insight as to how agriculture through precision farming implements IoT through the use of robots, drones, sensors, and computer imaging integrated with analytical tools for getting insights and monitoring the farms. Placement of physical equipment on farms monitors and records data, which is then used to get valuable insights.
   
   This had enabled us to get a broader view on how IoT allows devices across a farm to measure all kinds of data remotely and provide this information to the farmer in real time. IoT devices can gather information like soil moisture, chemical application, dam levels and livestock health – as well as monitor fences, vehicles and weather.\\
    

    
    \section*{Agreement, Pitfalls and Fallacies}
    %State what and why do you agree and disagree with the views presented in the paper. You can present this as a table, with the list of your agreements and disagreements on the key aspects of the paper (For example, you might feel that the conclusions drawn by the authors towards the claims made in the research might not agree to your views. Or else, you very well resonate with the thoughts of the author. In either case, you must list the same in brief phrases, and state the reason for each agreement/disagreement. The list can have as many as your understanding permits).
    
    \begin{itemize}
     The agreement would likely outline the specific tasks and responsibilities of the parties involved in the project, as well as any intellectual property rights or confidentiality agreements.\\
     
     By adopting IoT in the agricultural sector we get numerous benefits, but still, there are challenges faced by IoT in agricultural sectors. The biggest challenges faced by IoT in the agricultural sector are lack of information, high adoption costs, and security concerns, etc. Most of the farmers are not aware of the implementation of IoT in agriculture. Major problem is that some of them are opposed to new ideas and they do not want to adopt even if it provides numerous benefits. The best thing that can be done to raise awareness of IoT’s impact is to demonstrate farmers the use of IoT devices like drones, sensors and other technologies and they could provide them ease at work and accompanied by real-world examples.High Cost: Equipment needed to implement IoT in agriculture is expensive. 
     
     However sensors are the least expensive component, yet outfitting all of the farmers’ fields to be with them would cost more than a thousand dollars. Automated machinery cost more than manually operated machinery as they include cost for farm management software and cloud access to record data. To earn higher profits, it is significant for farmers to invest in these technologies however it would be difficult for them to make the initial investment to set up IoT technology at their farms.\\ 
     3. Lack of Security: Since IoT devices interact with older equipment they have access to the internet connection, there is no guarantee that they would be able to access drone mapping data or sensor readouts by taking benefit of public connection. An enormous amount of data is collected by IoT agricultural systems which is difficult to protect. Someone can have unauthorized access IoT providers database and could steal and manipulate the data.

    \end{itemize}
    
\end{multicols}
\end{document}

